\documentclass[10pt,landscape]{article}
\usepackage{multicol}
\usepackage{calc}
\usepackage{ifthen}
\usepackage[landscape]{geometry}
\usepackage[utf8]{inputenc}


% To make this come out properly in landscape mode, do one of the following
% 1.
%  pdflatex latexsheet.tex
%
% 2.
%  latex latexsheet.tex
%  dvips -P pdf  -t landscape latexsheet.dvi
%  ps2pdf latexsheet.ps


% If you're reading this, be prepared for confusion.  Making this was
% a learning experience for me, and it shows.  Much of the placement
% was hacked in; if you make it better, let me know...


% 2008-04
% Changed page margin code to use the geometry package. Also added code for
% conditional page margins, depending on paper size. Thanks to Uwe Ziegenhagen
% for the suggestions.

% 2006-08
% Made changes based on suggestions from Gene Cooperman. <gene at ccs.neu.edu>


% To Do:
% \listoffigures \listoftables
% \setcounter{secnumdepth}{0}


% This sets page margins to .5 inch if using letter paper, and to 1cm
% if using A4 paper. (This probably isn't strictly necessary.)
% If using another size paper, use default 1cm margins.
\ifthenelse{\lengthtest { \paperwidth = 11in}}
	{ \geometry{top=.5in,left=.5in,right=.5in,bottom=.5in} }
	{\ifthenelse{ \lengthtest{ \paperwidth = 297mm}}
		{\geometry{top=1cm,left=1cm,right=1cm,bottom=1cm} }
		{\geometry{top=1cm,left=1cm,right=1cm,bottom=1cm} }
	}

% Turn off header and footer
\pagestyle{empty}
 

% Redefine section commands to use less space
\makeatletter
\renewcommand{\section}{\@startsection{section}{1}{0mm}%
                                {-1ex plus -.5ex minus -.2ex}%
                                {0.5ex plus .2ex}%x
                                {\normalfont\large\bfseries}}
\renewcommand{\subsection}{\@startsection{subsection}{2}{0mm}%
                                {-1explus -.5ex minus -.2ex}%
                                {0.5ex plus .2ex}%
                                {\normalfont\normalsize\bfseries}}
\renewcommand{\subsubsection}{\@startsection{subsubsection}{3}{0mm}%
                                {-1ex plus -.5ex minus -.2ex}%
                                {1ex plus .2ex}%
                                {\normalfont\small\bfseries}}
\makeatother

% Define BibTeX command
\def\BibTeX{{\rm B\kern-.05em{\sc i\kern-.025em b}\kern-.08em
    T\kern-.1667em\lower.7ex\hbox{E}\kern-.125emX}}

% Don't print section numbers
\setcounter{secnumdepth}{0}


\setlength{\parindent}{0pt}
\setlength{\parskip}{0pt plus 0.5ex}
\usepackage{amsfonts}
\usepackage{mathrsfs}
\usepackage[intlimits]{amsmath}
\usepackage{stmaryrd}
\usepackage{relsize}
\usepackage{etoolbox}
% \renewcommand*\ttdefault{cmvtt}
% \renewcommand*\familydefault{\ttdefault} %% Only if the base font of the document is to be typewriter style
% \usepackage[OT1]{fontenc}
\usepackage[english]{babel}
\usepackage[shortlabels]{enumitem}
\usepackage{mathtools}
\usepackage{amssymb}
\usepackage{stmaryrd}
\usepackage{cancel}
\usepackage{mdframed}
\usepackage{framed}{}
\usepackage{tablefootnote} 
\usepackage{listings}
\usepackage{amsthm}
\usepackage[dvipsnames]{xcolor}
\usepackage{etoolbox}
\usepackage[all]{xy}
\usepackage{tikz}
\usepackage{thmtools}
%\usepackage{mathpazo}
\usepackage{stmaryrd}
%\usepackage{stix}
%\let\Hermaphrodite\relax
\usepackage{mathabx}
\let\Sun\relax
\let\Moon\relax
\let\Mercury\relax
\let\Venus\relax
\let\Earth\relax
\let\Mars\relax
\let\Jupiter\relax
\let\Saturn\relax
\let\Uranus\relax
\let\Neptune\relax
\let\Pluto\relax
\let\Gemini\relax
\let\Leo\relax
\let\Libra\relax
\let\Scorpio\relax
\let\Aries\relax
\let\Taurus\relax
\usepackage{marvosym}
\usepackage[
   pdfpagelabels=true,
   pdftitle={Geometry and Topology for Data Analysis},
   pdfauthor={Min Tang},
 ]{hyperref}
\usepackage{bookmark}
\usepackage[usenames,dvipsnames]{pstricks}
\usepackage{epsfig}
\usepackage{pst-grad} % For gradients
\usepackage{pst-plot} % For axes
\usepackage[space]{grffile} % For spaces in paths
\usepackage{etoolbox} % For spaces in paths
\makeatletter % For spaces in paths
\patchcmd\Gread@eps{\@inputcheck#1 }{\@inputcheck"#1"\relax}{}{}
\makeatother

\usetikzlibrary{cd}
\usetikzlibrary{calc}
\theoremstyle{definition}
\newtheorem{definition}{Def}
\newtheorem{defi}[definition]{Def}
\newtheorem{problem}[definition]{Problem}
\newtheorem{example}[definition]{Exp}
\newtheorem{cor}[definition]{Cor}
\newtheorem{alg}[definition]{Alg}

\theoremstyle{theorem}
\newtheorem{theorem}[definition]{Thm}
\newtheorem{prop}[definition]{Prop}
\newtheorem{lemma}[definition]{Lem}
%\newtheorem{claim}[definition]{Claim}
%\newtheorem{corollary}[definition]{Corollary}
%\theoremstyle{definition}
\newtheorem{rem}[definition]{Rem}
\newtheorem{remark}[definition]{Rem}

%\AfterEndEnvironment{definition}{\noindent\ignorespaces}
%\AfterEndEnvironment{example}{\noindent\ignorespaces}
%\AfterEndEnvironment{theorem}{\noindent\ignorespaces}
%%\AfterEndEnvironment{satz}{\noindent\ignorespaces}
%\AfterEndEnvironment{corollary}{\noindent\ignorespaces}
%\AfterEndEnvironment{remark}{\noindent\ignorespaces}
%\AfterEndEnvironment{remark'}{\noindent\ignorespaces}
%\AfterEndEnvironment{proposition}{\noindent\ignorespaces}
%\AfterEndEnvironment{proof}{\noindent\ignorespaces}
\let\existstemp\exists
\let\foralltemp\forall
\newcommand{\tikzmark}[1]{\tikz[overlay,remember picture] \node (#1) {};}
\newcommand{\vsubset}{\rotatebox[origin=c]{90}{$\subset$}}
\newcommand{\vphi}{\phi}
\newcommand{\ol}{\overline}
%Differentiation
\newcommand{\D}{\, \mathrm{d} }
%Bold Symbols
\newcommand{\R}{\mathbb{R}}
\newcommand{\C}{\mathbb{C}}
\newcommand{\N}{\mathbb{N}}
\newcommand{\Q}{\mathbb{Q}}
%Calligraphic Symbols
\newcommand{\ZZ}{\mathcal{Z}}
\newcommand{\rspace}{\vspace{-1.1pc} }
\newcommand{\II}{\mathcal{I}}
\newcommand{\FF}{\mathcal{F}}
\newcommand{\QQ}{\mathcal{Q}}
\newcommand{\EE}{\mathcal{E}}
\newcommand{\PP}{\mathcal{P}}
\newcommand{\TT}{\mathcal{T}}
\newcommand{\MM}{\mathcal{M}}
\newcommand{\HH}{\mathscr{H}}
\newcommand{\RR}{\mathscr{R}}
\newcommand{\BB}{\mathscr{B}}
\newcommand{\CC}{\mathscr{C}}
\newcommand{\DD}{\mathscr{D}}
\newcommand{\LL}{\mathscr{L}}
\newcommand{\NN}{\mathscr{N}}
\renewcommand{\AA}{\mathscr{A}}
\newcommand{\myfont}{\fontfamily{ptm}\selectfont }
% Operator names
\newcommand{\id}{\operatorname{id}}
\newcommand{\Hom}{\operatorname{Hom}}
\newcommand{\del}{\partial}
\newcommand{\GL}{\operatorname{GL}}
\newcommand{\vol}{\operatorname{vol}}
\newcommand{\Var}{\operatorname{Var}} 
\newcommand{\Cov}{\operatorname{Cov}}
\newcommand{\End}{\operatorname{End}}
\newcommand{\SL}{\operatorname{SL}}
\newcommand{\Aff}{\operatorname{Aff}}
\newcommand{\Isom}{\operatorname{Isom}}
\newcommand{\Trans}{\operatorname{Trans}}
\newcommand{\Bild}{\begin{tiny}(Bild hier)\end{tiny}}
\newcommand{\Ueb}{\begin{tiny}\textbf{(Ü)}\end{tiny}}
\newcommand{\te}{\text}
\newcommand{\comment}[1]{}
\newcommand{\drawaline}{\rule{0.325\textwidth}{0.2pt}}
%\renewcommand{\def}{\definition}
\renewcommand{\ker}{\operatorname{ker}}
\renewcommand*{\forall}{\foralltemp\mkern2mu}
\renewcommand{\emptyset}{\varnothing}
\renewcommand{\Re}{\operatorname{Re}}
\renewcommand{\O}{\operatorname{O}}
\renewcommand{\Im}{\operatorname{im}}
%\renewcommand{\qedsymbol}{$\blacksquare$}
\renewcommand{\phi}{\varphi}
\makeatletter 
\AfterEndEnvironment{mdframed}{%
 \tfn@tablefootnoteprintout% 
 \gdef\tfn@fnt{0}% 
}
\DeclareMathOperator{\vertex}{Vert}
\usepackage{xpatch}
\makeatletter
%\AtBeginDocument{\xpatchcmd{\@thm}{\thm@headpunct{.}}{\thm@headpunct{}}{}{}}
%\numberwithin{equation}{section}
%\numberwithin{equation}{section}
%\numberwithin{definition}{section}
%\numberwithin{theorem}{section}
%\numberwithin{lemma}{section}
%\numberwithin{prop}{section}
%\numberwithin{corollary}{section}
%\numberwithin{problem}{section}
%\numberwithin{example}{section}
%\numberwithin{remark}{section}
%\numberwithin{claim}{section}
\setlist[enumerate,1]{label={(\roman*)}}
\linespread{1.1}

% -----------------------------------------------------------------------

\usepackage{mathtools}

\begin{document}

\raggedright
\footnotesize
\begin{multicols*}{3}


% multicol parameters
% These lengths are set only within the two main columns
%\setlength{\columnseprule}{0.25pt}
\setlength{\premulticols}{1pt}
\setlength{\postmulticols}{1pt}
\setlength{\multicolsep}{1pt}
\setlength{\columnsep}{2pt}

%\begin{center}
%     \normalsize{\textbf{Geometry and Topology for Data Analysis}} \\
%\end{center}

%\section{Simplicial complexes}
\section{Abstract simplicial complexes}
\begin{definition}[Abstract simplicial complex]
An \emph{abstract simplicial complex} is a collection $K$ of non-empty finite sets $\emptyset \neq S \in K$ such that every nonempty subset of $S$ is also contained in $K$, i.e. $\emptyset \neq T \subseteq S \implies T \in K$.
\end{definition}
\rspace
\begin{definition}[Simplex]
Sets in $K$ are called \emph{simplices}.
\end{definition}
\rspace
\begin{definition}[Vertex]
An element $v\in S$ of a simplex $S\in K$ is a \emph{vertex} (plural: vertices).
\end{definition}
\rspace
\begin{defi}[Dimension of a simplex and of a complex]\ 
\begin{itemize}[itemsep=-2pt]
\item \vspace{-2pt} The \emph{dimension} of a \emph{simplex} $S\in K$ is $ |S| -1$.  
\item The \emph{dimension} of the \emph{complex} $K$ is $\sup_{S\in K}\dim S$. 
\end{itemize}
\end{defi}
\rspace
\begin{defi}[Face]
A \emph{face} is a subset of a simplex, \emph{proper} if it is a strict subset.
\end{defi}
\rspace
\begin{defi}[Coface]
A \emph{coface} is a superset of a simplex which is in the complex.
\end{defi}
\rspace
\begin{rem}
$\exists$ a bijection between the vertex set and 0-simplices.
\end{rem}
\rspace
\begin{defi}[Full subcomplex]
A subcomplex $L$ is full if every simplex in $K$ whose vertices are in $L$ belongs to $L$, i.e. $S\subseteq \operatorname{Vert}L, S\in K \implies S\in L$.
\end{defi}
\rspace
\drawaline
\vspace{-0.4pc}
\subsection{Vietoris-Rips complex}
{\myfont An abstract simplicial complex that can be defined from any metric space $X$ and distance $t$ by forming a simplex for every finite set of points that has diameter at most $t$:}
\vspace{-0.4pc}
\begin{definition}[Vietoris-Rips]
A \emph{Vietoris-Rips complex} of a metric space $(X,d)$ at scale $t$ is defined as 
\begin{equation*}
\operatorname{Rips}_t(X):= \{ \emptyset \neq Q \subseteq X \mid \operatorname{diam} Q \leq t \},
\end{equation*}
where $\operatorname{diam} Q := \sup_{x,y \in Q} d(x,y)$.
\end{definition}
\rspace\drawaline
\vspace{-1.5pc}
\begin{defi}[Closure]
The \emph{closure} of a  simplex is the collection of its faces, (is the smallest simplicial complex that contains all its faces).
%The \textbf{closure} of a simplex $S \in K$ contains all its faces:
%\begin{equation*}
%\operatorname{Cl} \ (S) := \{ R\in K \mid R \subseteq S\}.
%\end{equation*}
\end{defi}
\rspace
\begin{defi}[Star]
The \emph{star} of a simplex is the collection of its cofaces.
\end{defi}
\rspace
\begin{rem}
$\operatorname{St}(S) = \bigcap_{v \in S} \operatorname{St} \{v\}.$
\end{rem}
\rspace
\begin{defi}[Link] The \emph{link} of $S$ is defined as  \vspace{-0.4pc}
$$\operatorname{Lk} \ (S) := \{P \in K \mid P\cup S \in K , P \cap S = \emptyset\}.$$
\begin{center}
\includegraphics[scale=0.05]{./abb/2880px-Simplicial_complex_link.svg.png}
\end{center}
\end{defi}
\rspace
\begin{definition}[Join]
For two \emph{augmented} simplicial complex $K$ and $L$, $\vertex K \cap \vertex L = \emptyset$, their \emph{join} is a simplicial complex defined as
\begin{equation*}
K * L := \{S\cup T \mid S\in K, T\in L \}.
\end{equation*}
\end{definition}
\rspace
\begin{rem}
$\operatorname{Lk} S * \operatorname{Cl} S = \operatorname{Cl} (\operatorname{St} S).$
\end{rem}
\rspace%\columnbreak
\begin{definition}[Cone, Apex, Base]
For $v \notin \vertex K$, $\{\{ v\}\} * K$ is a \emph{cone} with \emph{apex} $v$ and \emph{base} $K$. 
%\begin{center}
%\includegraphics[scale=0.06]{./abb/Pyramid.svg.png}
%\end{center}
\end{definition}
\rspace \columnbreak
\begin{definition}[Suspension]
For $v,w \notin \vertex K$, $\{\{v\},\{w\}\} * K$ is a \emph{suspension} of $K$.
\end{definition}\vspace{-0.6pc}
{\myfont A \emph{flag complex} is an abstract simplicial complex such that\emph{ every set of vertices in which all pairs are simplices in the complex} is also itself a simplex in the complex.}\vspace{-0.6pc}
\begin{definition}[Flag complex]
Given a finite poset (partially order set) $P=(P,\leq)$, its \emph{flag complex} (or order complex) is the Cplx that has $P$ as vertices, and chains as simplices.
\end{definition}
\rspace
\begin{example}[Derived subdivision]
$\operatorname{Pos}K:=(K,\subseteq)$ is called the \emph{face poset} of $K$. Its flag complex is called the \emph{derived subdivision} of $K$, denoted by $\operatorname{Sd}K=\operatorname{Flag}(\operatorname{Pos}K)$.
\end{example}
\rspace
\drawaline
\vspace{-0.4pc}
\subsection{Maps between simplicial complexes}
Let $\mathcal{A}, \mathcal{B}$ be abstract simplicial complexes. \vspace{-0.5pc}
\begin{definition}[Vertex map] A map $\phi: \vertex \mathcal{A} \to \vertex \mathcal{B}$ between vertex sets is a \emph{vertex map} if the image of the vertices of a simplex always span a simplex, i.e.
$ S\in \mathcal{A} \implies \phi(S) \in \mathcal {B}.$
\end{definition}
\rspace
\begin{defi}[Abstract simplicial map]
The induced map 
$ f:\mathcal{A} \to \mathcal{B}, S \mapsto \phi(S)$
is an \emph{abstract simplicial map}.
\end{defi}
\rspace
\begin{defi}[Simplicial isomorphism]
If $\phi$ is a bijection and $\phi^{-1}$ is also a vertex map, then $f$ is a \emph{ simplicial isomorphism}.
\end{defi}
\rspace
\drawaline
\vspace{-0.4pc}
\section{Geometric simplicial complexes}
Let $V=\{v_0,...,v_k\}\subset \R^d$ be a finite point set. \vspace{-0.5pc}
\begin{definition}[Affine combination]
 A point $\sum_{v\in V} \lambda_v v$ with $\sum_{v\in V}\lambda_v =1$ is an \emph{affine combination} of $V$.
\end{definition}
\rspace
\begin{defi}[Affine hull]
The collection of affine combinations is the \emph{affine hull}, denoted by $\operatorname{aff} V$.
\end{defi}
\rspace
\begin{defi}[Affinely independent]The points in $V$ are \emph{affinely independent} if the coefficients of any affine combination are unique. In this case  $\operatorname{aff} V$ is a $k$-dim affine subspace.
\end{defi}
\rspace
\begin{definition}[Convex combination] 
An affine combination with $\lambda_v \geq 0 $ is a \emph{convex combination}. 
\end{definition}
\rspace
\begin{defi}[Convex hull]
The \emph{convex hull} is the collection of convex combinations, denoted by $\operatorname{conv} V$.
\end{defi}
\rspace
\drawaline
\vspace{-1.5pc}
\begin{definition}[General linear position]
$V$ is in \emph{general linear position} if any subset of at most $d+1$ points is affinely independent.
\end{definition}
\rspace
\begin{example}[Moment curve]
The moment curve $$X = \{(t,t^2,...,t^d)\} \mid t \in \R\} \subset \R^d$$ is in general linear position.
\end{example}
\rspace
\begin{rem}
Any finite $V\subset \R^d$ can be perturbed to general position, arbitrarily close to $X$.
\end{rem}
\rspace
\drawaline\\
%\columnbreak
Let $V$ be affinely independent. \vspace{-0.5pc}
\begin{defi}[Geometric $k$-simplex]
A \emph{geometric $k$-simplex} $\sigma = \operatorname{conv} V$ is the convex hull of $k+1$ affinely independent points. $k$ is the \emph{dimension} of the simplex.
\end{defi}
\rspace
\begin{defi}[Vertex]
We say $\sigma$ is spanned by $V$. The points in $V$ are called vertices, $V=\operatorname{Vert} \sigma$.
\end{defi}
\rspace
\begin{rem}
\emph{Vertex}: 0-simplex, \emph{edge}: 1-simplex, \emph{triangle}: 2-simplex, and \emph{tetrahedron}: 3-simplex.
\end{rem}
\rspace
\begin{defi}[Face]
A \emph{(proper) face} is a simplex spanned by a non-empty (proper) subset of $V$.
\end{defi}
\rspace
\begin{defi}[Coface]
$\tau$ is coface of $\sigma$ $:\iff$ $\sigma$ is face of $\tau$
\end{defi}
\rspace
\begin{defi}[Boundary]
The \emph{boundary} of $\sigma$ ($\operatorname{bd} \sigma$) is the union of all its proper faces, or equivalently, all faces of codimension 1.
\end{defi}
\rspace
\begin{rem}
The boundary of a vertex is $\emptyset$.
\end{rem}
\rspace
\begin{defi}[Interior]
$\operatorname{int} \sigma = \sigma \setminus \operatorname{bd} \sigma$.
\end{defi}
\vspace{-1pc}
\drawaline\\ 
{\myfont We are interested in sets of simplices that are closed under taking faces and that have no improper intersections:}\vspace{-0.6pc}
\begin{definition}[Geometric simplicial complex] A \emph{geometric simplicial complex} $K$ in $\R^d$ is a collection of simplices that 
\begin{enumerate}[itemsep = -7pt]
\item is closed under the face relation: \vspace{-0.3pc}
$$ \sigma \in K \ \ \text{and}\ \  \tau \text{ face of } \sigma \implies \tau \in K,$$
\item intersects only in common faces: for every $\sigma, \tau \in K$ with $\sigma \cap \tau \neq \emptyset$, it is a face of both $\sigma, \tau$.
\end{enumerate}
\end{definition}
\rspace
\begin{definition}[Dimension]
The \emph{dimension} of $K$ is the supremum dimension of its simplices.
\end{definition}
\rspace
\begin{rem}
Condition (ii) is equivalent to: (ii') simplices have disjoint interiors:
$\sigma \neq \tau \in K \implies \operatorname{int}  \sigma \cap \operatorname{int} \tau = \emptyset.$
\end{rem}
\rspace
\drawaline
\vspace{-1.5pc}
\begin{definition}[Underlying Space] The \emph{underlying space} of $K$ (or \emph{polyhedron}), denoted as $|K|$, is the union of its simplices together with the \emph{coherent topology,} which can be expressed using closed sets:  $U \subseteq |K|$ closed if and only if $U\cap \sigma$ is closed in subspace topology of $\sigma$ $\forall \sigma \in K$.
\end{definition}
\rspace
\begin{definition}[Triangulation]
A \emph{triangulation} of a topological space $X$ is  a homeomorphism between $X$ and $|K|$ for some simplicial complex $K$. In this case, $X$ is \emph{triangulable}.
\end{definition}
\rspace
%\drawaline
%\vspace{-1.5pc}
%\begin{definition}[Vertex scheme]
%The abstract simplicial complex induced by the vertex sets of the simplices in $K$ is called the \emph{vertex scheme} of $K$. 
%\end{definition}
%\rspace
%\begin{defi}[Geometric realization]
%If the vertex scheme of $K$ is isomorphic to some abstract simplicial complex $\mathcal{A}$, then $K$ is a \emph{geometric realization} of $\mathcal{A}$.
%\end{defi}
%\rspace
\drawaline
%\vspace{-1.5pc}
%\begin{definition}[Underlying Space] The \emph{underlying space} of $K$ (or \emph{polyhedron}), denoted as $|K|$, is the union of its simplices together with the \emph{coherent topology,} which can be expressed using closed sets:  $U \subseteq |K|$ closed if and only if $U\cap \sigma$ is closed in subspace topology of $\sigma$ $\forall \sigma \in K$.
%\end{definition}
%\rspace
%\drawaline
\vspace{-0.5pc}
\subsection{Simplicial maps}
Let $K,L$ be geometric simplicial complexes. 
\vspace{-0.5pc}
\begin{definition}[Simplicial map]A \emph{simplicial map} is a continuous map $f: |K| \to |L|$ mapping each simplex of $K$ affinely onto some simplex in $L$. We also call $f: K \to L$ a \emph{simplicial map}. 
\end{definition}
\rspace
\begin{rem}%[Simplicial isormorphism]
The restriction to the vertices of $K$ already determines $f$ uniquely.
\end{rem}
\rspace
\begin{definition}[Simplicial isomorphism] If $f$ has a simplicial inverse, it is a \emph{simplicial isomorphism}.
\end{definition}
\rspace
\drawaline
\vspace{-1.5pc}
\begin{definition}[Vertex scheme]
The abstract simplicial complex induced by the vertex sets of the simplices in $K$ is called the \emph{vertex scheme} of $K$. 
\end{definition}
\rspace
\begin{defi}[Geometric realization]
If the vertex scheme of $K$ is isomorphic to some abstract simplicial complex $\mathcal{A}$, then $K$ is a \emph{geometric realization} of $\mathcal{A}$.
\end{defi}
\rspace
\begin{rem}
The geometric  realization of an abstract simplicial complex is unique up to simplicial isomorphism.
\end{rem}
\rspace
\begin{remark}[Geometric Realization Theorem] 
An $d$-dim abstract simplicial complex with $n$ vertices has geometric realizations in $\R^n$, $\R^{n-1}$, and in $\R^{2d+1}$.
\end{remark}
\rspace
%\drawaline\\
%{\myfont A simplicial complex $L$ is a \emph{subdivision} of another simplicial complex $K$ if they have the same underlying space, i.e. $|L|$ = $|K|$, and every simplex in $L$ is contained in a simplex in $K$.} \vspace{-0.5pc}
%\begin{example}[Barycentric subdivision]
%The \emph{barycentric subdivision} of $K$, denoted as $\operatorname{Sd} K$, is defined as follows:\\
%%\begin{enumerate}
%(i) The vertices are the \emph{barycenters} of simplices in $K$:
%$$\vertex \operatorname{Sd} K = \bigg\{ z(\sigma) = \sum_{i=0}^d \frac{1}{d+1} v_i \, \bigg|\, \sigma = \operatorname{conv} \{v_0,...,v_d\}\in K \bigg\},$$
%(ii) The $k$-simplices of $\operatorname{Sd} K$ are spanned by the barycenters:
%$$ \{z(\sigma_0), z(\sigma_1),...,z(\sigma_k) \},$$
%where $\sigma_0 \subsetneq \sigma_1 \subsetneq ... \subsetneq \sigma_k$ is a flag of proper faces in $K$.
%%\end{enumerate}
%\begin{center}
%\includegraphics[scale=0.7]{./abb/barycentric.pdf}
%\end{center}
%\end{example}
\drawaline \vspace{-0.5pc}
\subsubsection{Gluing Constructions}
%Let $X$ be a topological space, $\sim$ an equivalence relation on $X$, $Y= X/ \sim$ the equivalence classes.  \vspace{-0.5pc}
%\begin{definition}[Quotient map]
%The surjection $q: X \to Y, x\mapsto [x]$ is the \emph{canonical quotient map}.
%\end{definition}
%\rspace
%\begin{defi}[Quotient topology]The \emph{quotient topology} on $Y$ induced by $\sim$ is the collection $\{ U \subseteq Y \mid q^{-1} (U) \text{ open in } X\}$. 
%\end{defi}
%\rspace
%\begin{defi}[Quotient space]$Y$ is the \emph{quotient space} of $X$ under $\sim$.
%\end{defi}
%\rspace
%\begin{rem}
%In general, any continuous surjection $f:X\to Y$ with the property that $U\subseteq Y$ is open iff $f^{-1}(U) \subseteq X$ is open is called a \emph{quotient map}.
%\end{rem}
\vspace{-0.2pc}
Let $L$ be an abstract simplicial complex and $f:\vertex L \to V$ a surjective map to some set $V$ of vertex labels. \vspace{-0.5pc}
\begin{definition}[Pasting complex, map] The collection $K=\{f (\sigma) \mid \sigma \in L\} $ is an abstract simplicial complex with vertices $\vertex K = V$, called the \emph{pasting complex} for $f$.
\end{definition}
\rspace
\begin{defi}[Pasting map]
The \emph{pasting map} for $f$ is the simplicial map induced by $f$.
\end{defi}
\vspace{-0.5pc}
%Then:
%\begin{itemize}[itemsep= 0pt]
%\item $K=\{f (\sigma) \mid \sigma \in L\} $ is an abstract simplicial complex with vertices $\vertex K = V$, called the \textbf{pasting complex} for $f$.
%\item The map $f$ induces a simplicial map $g:L \to K$, called the \textbf{pasting map} for $f$.
%\item The corresponding map $|L| \to |K|$ is a quotient map. 
%\end{itemize}
{\myfont  We want to identify only disjoint simplices in $L$ and the identification should preserve dimension.
We now phrase criterion to exclude unintended gluing of simplices.} \vspace{-0.5pc}
\begin{rem}
Assume that the vertices in $L$ with the same label have disjoint closed vertex stars: for any $v \neq w \in \vertex L$ with $f(v)=f(w)$, we have $\operatorname{Cl}\, \operatorname{St} v \cap \operatorname{Cl}\, \operatorname{St} w = \emptyset$. Then,
\begin{enumerate}[itemsep = 0pt]
\item the induced pasting map preserves dimension of simplices: $\dim g(\sigma)= \dim \sigma$ for all $\sigma \in L$, and
\item only disjoint simplices are identified: if $\sigma, \tau \in L$ with $g(\sigma) = g(\tau)$, then $\sigma \cap \tau = \emptyset$.
\end{enumerate}
\end{rem}
\rspace
\drawaline \vspace{-0.5pc}
\subsection{Homotopy} 
Let $f,g: X\to Y$ be two continuous maps  between topological spaces.\vspace{-0.5pc}
\begin{definition}[Homotopic] $f$ and $g$ are \emph{homotopic} ($f\simeq g$) if there exists a continuous deformation of $f$ into $g$, i.e. a continuous map $H:X\times [0,1] \to Y$ with $H(x,0)=f(x)$ and $H(x,1) = y(x)$ for all $x\in X$.
\end{definition}
\rspace
\begin{defi}[Homotopy]
$H$ is called a \emph{homotopy}.
\end{defi}
\rspace
\begin{remark}
Homotopy defines an equivalence relation of continuous maps.
\end{remark}
\rspace
\drawaline \vspace{-0.5pc}
\subsection{Simplicial approximation}
{\myfont We show that by subdividing, we can approximate any continuous map between triangulable spaces by a simplicial map.} \\
Let $K,L$ be geometric simplicial complexes and let $g:|K|\to |L|$ be a continuous map.
\vspace{-0.5pc}
\begin{definition}[Simplicial approximation]
A simplicial map $f:K\to L$ is a \emph{simplicial approximation} to $g$, if for all $x \in |K|$ and $\tau \in L$, $g(x) \in \tau$ implies $f(x)\in \tau$.
\begin{center}
\vspace{-1.2pc}
\includegraphics[scale=0.55]{./abb/simpapprx.pdf}
\end{center}
\end{definition}
\rspace\vspace{-0.5pc}
\begin{remark}
$f$ and $g$ are homotopic.
\end{remark}
\rspace
\drawaline\\\vspace{-0.6pc}
%\vspace{-1.5pc}
\begin{defi}[Subdivision]
{A simplicial complex $L$ is a \emph{subdivision} of another simplicial complex $K$ if they have the same underlying space, i.e. $|L|$ = $|K|$, and every simplex in $L$ is contained in a simplex in $K$.}
\end{defi}
\rspace
\begin{example}[Barycentric subdivision]
The \emph{barycentric subdivision} of  $K$, denoted as $\operatorname{Sd} K$, is defined as follows:\\
(i) The vertices are the barycenters of simplices of $K$:\vspace{-0.35pc}
$${\scriptstyle \vertex \operatorname{Sd} K = \big\{ z(\sigma) = \sum_{i=0}^d \frac{1}{d+1} v_i \, \big|\, \sigma = \operatorname{conv} \{v_0,...,v_d\}\in K \big\},} \vspace{-0.39pc}$$ 
(ii) The $k$-simplices of $\operatorname{Sd} K$ are spanned by the barycenters:\vspace{-0.4pc}
$$ {\scriptstyle \{z(\sigma_0), z(\sigma_1),...,z(\sigma_k) \},}\vspace{-0.4pc}$$
where $\sigma_0 \subsetneq \sigma_1 \subsetneq ... \subsetneq \sigma_k$ is a \emph{chain} of proper faces in $K$.
\begin{center}
\vspace{-0.6pc}
\includegraphics[scale=0.6]{./abb/barycentric.pdf}
\vspace{-0.5pc}
\end{center}
\end{example}
\rspace
\drawaline\\\vspace{-0.6pc}
\begin{rem}
Every point $x\in |K|$ lies in the unique interior of a simplex $\sigma_x \in K$. All coefficients $\lambda_v$ in 
$x = \sum_{v\in \sigma_x} \lambda_v v$
are strictly positive.  
\end{rem}
\rspace
\begin{definition}[Barycentric coordinates]
For each vertex $v \in \vertex K$, define the coordinate function $b_v : |K|  \to [0,1]$ \vspace{-0.38pc}
\begin{equation*}
{\scriptsize
x \mapsto 
\begin{cases}
\lambda_v, \quad & \text{if }v \text{ is a vertex of  }\sigma_x, \vspace{-0.1pc} \\
0,\quad & \text{otherwise}.
\end{cases} }
\vspace{-0.38pc}
 \end{equation*}
The functions $b_v$ are the \emph{barycentric coordinates} of $K$.
 \end{definition}
 \rspace
 \begin{rem}
Barycentric coordinates are continuous.
 \end{rem}
 \rspace
  \begin{rem}
The unique simplex $\sigma_x \in K$ with $x \in \operatorname{int} \sigma_x$ is spanned by vertices
$
\{ v\in \vertex K \mid b_v(x) >0\}.
$
\end{rem}
\rspace
\drawaline\\\vspace{-0.7pc}
\begin{definition}[Open star]
The \emph{open star} of $\sigma \in K$ is 
$
\operatorname{st} \sigma = \bigcup_{\tau \in \operatorname{St} \sigma} \operatorname{int} \tau.
$
\end{definition}
\rspace
\begin{rem}
The open star is an open set: for a vertex $v$ we have
$\operatorname{st}(v)  = \{x \in |K| \mid b_v(x)>0\} = b_v^{-1}(0,\infty).$
\end{rem}
\rspace
\begin{remark}A vertex set $S$ spans a simplex $\tau$ iff their open star intersection is non-empty, i.e. $\bigcap_{v\in S} \operatorname{st}(v) \neq \emptyset,$ in which case we have
$
\operatorname{st}(\sigma) = \bigcap_{v\in S}\operatorname{st}(v). $
\end{remark}
\rspace\drawaline\\\vspace{-0.7pc}
\begin{definition}[Star condition]
$g: |K| \to |L|$ satisfies the \emph{star condition} if it maps open stars in $K$ into open stars in $L$, i.e.
$\forall v \in \vertex K \ \exists u \in \vertex L: g(\operatorname{st}v) \subseteq \operatorname{st} u.$
\end{definition}
\rspace
\begin{rem}
Choosing such a vertex $u_v$ for each vertex $v$ defines a vertex map $\phi: \vertex K \to \vertex L, v \mapsto u_v$, which extends to a simplicial approximation $f:K\to L$ of $g$.
\end{rem}
\rspace\drawaline\\\vspace{-0.7pc}
\begin{definition}[Mesh]
The \emph{mesh} of $K$ is the maximum length of its edges. It is also the maximum \emph{diameter} of any simplex.
\end{definition}
\rspace
\begin{rem}
Let $\delta$ be the mesh of $K$. Then the mesh of $\operatorname{Sd} K$ is at most $\frac{d}{d+1}\delta$, where $d$ is the dimension of $K$.
\end{rem}
\rspace
\begin{theorem}[Simplicial approximation]
Let $K,L$ be finite geometric simplicial complexes. Then $g$ has a simplicial approximation $f:|\operatorname{Sd}^n K| \to |L|$ for some $n\in \N$.
\end{theorem}
\rspace\drawaline\\\vspace{-0.5pc}
\section{Complexes from geometric point sets}
\subsection{Homotopy equivalence}
{\myfont Weaker than homeomorphism.} Let $X,Y$ be topological spaces.
\columnbreak
\begin{definition}[Homotopy equivalent]
Two spaces $X, Y$ are \emph{homotopy equivalent} ($X \simeq Y$) if $\exists$ continuous $f:X\to Y$, $g:Y\to X$ such that
$ f\circ g \simeq \operatorname{Id}_Y\ \text{and} \ g\circ f \simeq \operatorname{Id}_X$.
\end{definition}
\rspace
\begin{defi}[Homotopy inverse]
$g$ is a \emph{homotopy inverse} to $f$ and vice versa.
\end{defi}
\rspace
\begin{remark}
Homotopy equivalence defines an equivalence relation of topological spaces.
\end{remark}\vspace{-0.6pc}
{\myfont Consider $Y\subseteq X$. Then a continuous map $r:X \to Y$ is a \emph{retraction} if $r$ restricts to the identity map on $Y$. More generally:}\vspace{-0.5pc}
\begin{definition}[Retraction]
Given $\iota: Y \to X$ continuous, a continuous $r:X\to Y$
is a \emph{retraction} if $r\circ \iota =\id_Y$.
\end{definition}
\rspace
\begin{defi}[Deformation retraction]
The map $r:X\to Y$ is a \emph{deformation retraction} if $\iota\circ r$ is homotopic to $\id_X$.
\end{defi}
\rspace
\begin{rem}
A {deformation retraction} can also be defined as a homotopy between a retraction and the identity map on $X$.
\end{rem}
\rspace
\begin{defi}[Contractible]
Assume $\exists$ deformation retraction $X\to Y$. If $Y$ is a single point, then $X$ is \emph{contractible}.
\end{defi}
\rspace
\begin{remark}
Two spaces $X$ and $Y$ are homotopy equivalent iff $\exists Z$ s.t. $\exists$ deformation retractions $Z \to X$ and $Z \to Y$.
\end{remark}
\rspace
\drawaline \vspace{-0.5pc}
\subsection{Nerves} 
{\myfont We construct simplicial complexes by recording the intersection patterns of a collection of sets.} \vspace{-0.6pc}
\begin{definition}[Nerve]
Let ${F}$ be a finite collection of sets. The \emph{nerve} of ${F}$ consists of all subcollections whose sets have a non-empty common intersection,
$
\operatorname{Nrv} F = \{G\subseteq {F} \mid \bigcap G \neq \emptyset\}.
$
\begin{center}
\vspace{-0.8pc}
\includegraphics[scale=0.55]{./abb/nerve.pdf}
\vspace{-0.7pc}
\end{center}
\end{definition}
\rspace
\begin{rem}
$\operatorname{Nrv} F$ is always an abstract simplicial complex: if $\bigcap X \neq \emptyset$ and $Y\subseteq X$, then $\bigcap Y \neq \emptyset$ as well.
\end{rem}
\rspace
\begin{theorem}[Nerve theorem for compact convex covers]Let $X$ be compact.
Assuming that $F$ is a finite cover of $X$ by compact convex sets. Then $|\operatorname{Nrv} F| \simeq X$.
\end{theorem}
\rspace
\drawaline \vspace{-0.5pc}
\subsection{Čech complexes} 
{\myfont Consider the case where $F$ is a collection closed balls.} Let $X\subset \R^d$ be a \emph{finite point set}. Write $D_r(x) = \{y\in \R^d \mid \|y-x\|\leq r\}$.\vspace{-0.6pc}
\begin{definition}[Čech complex]
The Čech complex of $X$ for radius $r$ is defined as
$
\text{Čech}_r(X) = \{Q \subseteq X \mid \bigcap_{x\in Q}D_r(x)\neq \emptyset \}.
$
\begin{center}
\vspace{-0.8pc}
\includegraphics[scale=0.5]{./abb/cech.pdf}
\vspace{-0.8pc}
\end{center}
\end{definition}
\rspace \columnbreak
\begin{rem}
Čech complexes are abstract simplicial complexes. 
\end{rem}
\rspace
\begin{rem}
$\text{Čech}_r(X)$ is isomorphic to the never of the balls centered at the points in $X$.
\end{rem}
\rspace
\begin{rem}
If $X$ is compact, then for every $r\geq \operatorname{diam} X$, the Čech complex is the full simplex, i.e. $\operatorname{Cech}_r(X)=\operatorname{Cl} X$.
\end{rem}
\rspace
\begin{remark}
$Q=\{q_1,...,q_n\} \in \operatorname{Cech}_r(X)  \iff \exists z \in\bigcap_{q\in Q}D_r(q) $ $\iff Q\subseteq D_r(z)$ $\iff $ the smallest enclosing sphere of $Q$ has radius $\leq r$.
\end{remark}
\rspace
\drawaline\\\vspace{-0.5pc}
\subsubsection{Enclosing spheres and circumspheres}\vspace{-0.15pc}
Let $Q\subset \R^d$ be a finite point set and let $D$ be a $d$-ball with boundary (($d-1$)-sphere) $S$.\vspace{-0.6pc}
\begin{definition}[Enclosing sphere]$S$ is an \emph{enclosing sphere} of $Q$, if $Q\subset D$. In this case, $D$ is an \emph{enclosing disk} of $Q$.
\end{definition}
\rspace
\begin{definition}[Circumsphere] $S$ is \emph{circumsphere} of $Q$, if $Q\subset S$.
\end{definition}
\rspace
\begin{remark}
Affinely independent $Q$ has a circumsphere. 
\end{remark}
\rspace
\begin{rem}[Smallest circumsphere]If $|Q| \leq d$, then its circumsphere is not unique. But the smallest circumsphere $S(Q)$ is unique and has center on  the affine hull $\operatorname{aff} Q$.
\end{rem}
\rspace
\begin{remark}
Assume $B(Q,P)$ is the smallest circumsphere of $P$ that encloses $Q$ and $x\in Q$. If the sphere $B(Q \setminus \{x\},P)$  encloses $x$, then ${ B(Q,P) = B(Q \setminus \{x\},P)}$, and otherwise ${ B(Q,P) = B(Q \setminus \{x\},P \cup \{x\})}$.
\end{remark}
\rspace
\drawaline\\\vspace{-0.5pc}
\subsection{Voronoi domains and Delaunay complexes}
Let $X\subset \R^d$ be a finite point set. \vspace{-0.6pc}
\begin{defi}[Voronoi domain]
The \emph{voronoi domain} of $x\in X$ is the set of points $p$ in $\R^d$ that have $x$ as the nearest neighbor in $X$:
$
\operatorname{Vor}(x,X) = \{p\in\R^d \mid \|p-x\|\leq \|p-y\|\ \text{for all } y\in X\}.
$
%\begin{center} \vspace{-0.6pc}
%\includegraphics[scale=0.5]{./abb/voronoi.pdf}\vspace{-0.6pc}
%\end{center}
\end{defi}
\rspace
\begin{defi}[Delaunay complex]
The \emph{Delaunay complex} of $X$ is defined as
$\operatorname{Del} (X) := \{ Q\subseteq X  \mid \bigcap_{x\in Q}\operatorname{Vor}(x,X)\neq \emptyset\}$.
\begin{center}\vspace{-0.6pc}
\includegraphics[scale=0.55]{./abb/del.pdf} \vspace{-0.5pc}
\end{center}
\end{defi}
\rspace
\begin{rem}
The Delaunay complex is isomorphic to the nerve of the Voronoi domains.
\end{rem}
\rspace
\begin{rem}
$Q\in \operatorname{Del}(X) \implies \exists x \in \R^d, r>0: \|x-q\|=r \forall q\in Q.$
\end{rem}
\rspace
\begin{remark}
$\emptyset \neq Q \subseteq X$ is a simplex of $\operatorname{Del} (X)$ iff $Q$ has a {circumsphere} $S$ ($Q\subset S$) bounding an open ball $B$ that is empty of points in $X$: $X\cap B= \emptyset$.
\end{remark}
\rspace\drawaline\\\vspace{-0.7pc}
\begin{definition}[General spherical position]
$X$ is in \emph{general spherical position} if for every $Q\subseteq X$ of at most $d+1$ points, (i) $Q$ is affinely independent; and (ii) the smallest circumsphere of $Q$ contains no other points of $X$:
$S(Q) \cap (X\setminus Q) = \emptyset.$
\end{definition}
\rspace
\begin{theorem}
Let $X$ be in {general spherical position}. Then the  geometric simplices
$\operatorname{GeomDel}(X) = \{\operatorname{conv} Q \mid Q \in \operatorname{Del}(X)\}$
form a geometric simplicial complex. Its underlying space is the convex hull of the points $X$, i.e.
$|\operatorname{GeomDel(X)}| =\operatorname{conv} X.$
\end{theorem}
\rspace \columnbreak
\begin{defi}[Lifting]
$l:\R^d\to \R^{d+1}, x\mapsto (x,\|x\|^2)$.
\end{defi}
\rspace
\begin{lemma}
A $(d-1)$-sphere $S\in \R^d$ corresponds to a unique hyperplane $H= \operatorname{aff} l(S) \subseteq \R^{d+1}$ with $l(S)=H \cap \operatorname{im} l$ (an ellipsoid). $S$ is empty of $X$ iff all points in $l(X)$ lie in the upper halfspace bounded by $H$.
\end{lemma}
\rspace
\begin{cor}
If $X\subset \R^d$ is in \emph{general spherical position}, then $l(X)$ is in \emph{general linear position}.
\end{cor}
\rspace
\drawaline\\\vspace{-0.5pc}
\subsection{Delaunay Filtration}
\begin{definition}[Delaunay complex]
The \emph{Delaunay complex} of $X$ \emph{at radius} $r\geq 0$ is 
$
\operatorname{Del}_r(X) = \{ Q\subseteq X \mid \bigcap_{x\in Q} \operatorname{Vor}_r(x,X) \neq \emptyset \},
$
where
$\operatorname{Vor}_r(x,X)=D_r(x)\cap \operatorname{Vor}(x,X).$
\begin{center}\vspace{-0.53pc}
\includegraphics[scale=0.55]{./abb/alpha.pdf} \vspace{-0.34pc}
\end{center}
\end{definition}
\rspace
\begin{rem}
$\operatorname{Del}_r(X) \cong \operatorname{Nrv} {\{\operatorname{Vor}_r(x,X)\mid x\in X\}}$
\end{rem}
\rspace
\begin{rem}
For sufficiently large $r$: $\operatorname{Del}_r(X)= \operatorname{Del}(X)$.
\end{rem}
\rspace
\begin{remark} The Čech and Delaunay complexes for the same radius have homotopy equivalent geometric realizations:
$|\operatorname{Del}_r(X)| \simeq |\operatorname{Cech}_r(X)|\simeq \bigcup_{x\in X} D_r(x).$
\end{remark}
\rspace
\begin{remark}
$Q \subseteq X$ is a simplex of $\operatorname{Del}_r (X)$ iff the smallest circumsphere of $Q$ excluding $X$ has radius at most $r$.
\end{remark}
\rspace
\begin{rem}
$\operatorname{Del}_r(X) \neq \operatorname{Del}(X)\cap \operatorname{Cech}_r(X)$.
\end{rem}
\rspace\drawaline\\\vspace{-0.5pc}
\section{Homology}
Let $K$ be a finite simplicial complex. \vspace{-0.6pc}
\begin{definition}[Euler characteristic]
Let $k_i$ be the number of $i$-simplices. Then, the $\chi(K)=\sum_{i=0}^{\dim K}(-1)^i k_i$ is the \emph{Euler characteristic}.
\end{definition}
\rspace
\begin{rem}
If $|K| \simeq |L|$, then $\chi(K)=\chi(L)$.
\end{rem}
\rspace
\begin{definition}[$d$-chain]A \emph{$d$-chain} is a \emph{formal sum} of $d$-simplices
$\sum_{\sigma\in K_{(d)}} \lambda_\sigma \sigma$
with coefficients in  $\mathbb{Z}_2$.
\end{definition}
\rspace
\begin{defi}[Chain space]
The $d$-chains form a $\mathbb{Z}_2$-vector space $C_d(K)$, the \emph{chain space}.
\end{defi}
\rspace
\begin{rem}
A $d$-chain corresponds to a subset of $d$-simplices.
\end{rem}
\rspace
\begin{defi}[Boundary map]
 The \emph{boundary} $\del \sigma$ of a $d$-simplex $\sigma$ is the $(d-1)$-chain of facets (faces of codimension 1) of $\sigma$.  This extends linearly to a \emph{boundary map} $\del_d: C_d(K) \to C_{d-1}(K)$.
 \end{defi}
\rspace
\begin{defi}[$d$-boundary]
A $d$-chain $\gamma$ is called a \emph{$d$-boundary} if $\gamma = \del_{d+1}(\xi)$ for some $(d+1)$-chain $\xi$. 
\end{defi}
\rspace
\begin{rem}
The $d$-boundaries $B_d := \operatorname{im}\del_{d+1}$ form a subspace of $C_d(K)$, containing boundaries of $(d+1)$ chains.
\end{rem}
\rspace
\begin{defi}[$d$-cycle]A $d$-chain $\gamma$ is a $d$-cycle if $\del_d(\gamma)=0$. 
\end{defi}
\rspace
\begin{rem}
The $d$-cycles form a subspace $Z_d(K):=\operatorname{ker}\del_d$ of , containing chains with zero boundary.
\end{rem}
\rspace \columnbreak
\begin{rem}
Boundaries are cycles: $B_d(K)\subseteq Z_d(K)$. In other words, $\del_d\del_{d+1}(\sigma)=0$ for any $(d+1)$-chain $\sigma$.
\begin{center}\vspace{-0.53pc}
\includegraphics[scale=0.75]{./abb/chains.pdf} \vspace{-0.34pc}
\end{center}
 \end{rem}
\rspace
\begin{defi}[Homology]
The $d$-th \emph{homology (space)} is the $\mathbb{Z}_2$-vector space
$H_d(K) = Z_d(K) / B_d(K).$
\end{defi}
\rspace
\begin{defi}[Betti number]
The dimension $\beta_d(K)= \dim H_d(K)$ is the $d$-th \emph{Betti number}.
\end{defi}
\rspace
\begin{rem}
The betti numbers in dimension 0,1 and 2 count the number of connected components, tunnels, and voids of $K$.
\end{rem}
\rspace
\drawaline \vspace{-0.5pc}
\subsection{Induced maps}
{\myfont A simplicial map induces a map between homology groups.} Let $f:K\to L$ be a simplicial map.\vspace{-0.5pc}
\begin{defi}[Induced chain map]
The map $f$ induces a chain map $f_\#:C_d(K)\to C_d(L)$ by specifying it on the basis, the $d$-simplices of $K$, i.e. $\sigma \mapsto f(\sigma)$ if $f(\sigma)$ is a $d$-simplex, otherwise $0$.
\end{defi}
\rspace
\begin{lemma}
The map $f_\#$ induces a map $f_*:H_d(K)\to H_d(L)$ on homology.
\end{lemma}
\rspace
\begin{rem}
The homology is a \emph{functor} from simplicial complexes to vector spaces. In particular, commutative diagrams are preserved. \vspace{-0.4pc}
\begin{equation*}
	\begin{tikzcd}
	K \arrow{r}{f} \arrow{rd}[below]{g\circ f\ \ \ \ } & L \arrow{d}[right]{g}\\
					 & M
	\end{tikzcd}
	\quad   \implies \quad 
		\begin{tikzcd}
	H_*(K) \arrow{r}{f_*} \arrow{rd}[below]{g_*\circ f_*\ \ \ \ } & L \arrow{d}[right]{g_*}\\
					 & M
	\end{tikzcd}
	\end{equation*}
\end{rem}
\rspace
\begin{theorem}
Let $f$ be a simplicial map and a homotopy equivalence. Then the induced homomorphism $H_*(f):H_*(K)\to H_*(L)$ is an isomorphism.
\end{theorem}
\rspace
\drawaline \vspace{-0.5pc}
\subsection{Computing homology}
{\myfont We want to find compatible basis for the subspaces.} \vspace{-0.5pc}
\begin{defi}[Pivot]
Let $M$ be a matrix. Then the \emph{pivot index} of a column is the index of the last non-zero entry.
\end{defi}
\rspace
\begin{defi}[Reduced]
$M$ is \emph{reduced} if the pivots of non-zero columns are distinct.
\end{defi}
\rspace
\begin{alg}[Matrix reduction] Let $D$ be the boundary matrix and $V$ the identity matrix encoding column operations. As long as $D$ is not reduced, we a left column to a right column  of $D$ and $V$ simultaneously. Output: $D$ and $V$.
\end{alg}
\rspace
\begin{rem} The columns $\{ D_j \neq 0\}$ form a basis of $B_*(K)$.
\end{rem}
\rspace
\begin{rem}
The columns $\{V_i \mid D_i=0\}$ form a basis of $Z_*(K)$. 
 \end{rem}
\rspace
\begin{rem}
The columns $\{V_i+B_d(K) \mid D_i =0, i \notin \text{pivots } D\}$ form a basis for $H_*(K)$.
\end{rem}
\end{multicols*}
\end{document}
