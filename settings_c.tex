\documentclass[10pt,landscape]{article}
\usepackage{multicol}
\usepackage{calc}
\usepackage{ifthen}
\usepackage[landscape]{geometry}
\usepackage[utf8]{inputenc}


% To make this come out properly in landscape mode, do one of the following
% 1.
%  pdflatex latexsheet.tex
%
% 2.
%  latex latexsheet.tex
%  dvips -P pdf  -t landscape latexsheet.dvi
%  ps2pdf latexsheet.ps


% If you're reading this, be prepared for confusion.  Making this was
% a learning experience for me, and it shows.  Much of the placement
% was hacked in; if you make it better, let me know...


% 2008-04
% Changed page margin code to use the geometry package. Also added code for
% conditional page margins, depending on paper size. Thanks to Uwe Ziegenhagen
% for the suggestions.

% 2006-08
% Made changes based on suggestions from Gene Cooperman. <gene at ccs.neu.edu>


% To Do:
% \listoffigures \listoftables
% \setcounter{secnumdepth}{0}


% This sets page margins to .5 inch if using letter paper, and to 1cm
% if using A4 paper. (This probably isn't strictly necessary.)
% If using another size paper, use default 1cm margins.
\ifthenelse{\lengthtest { \paperwidth = 11in}}
	{ \geometry{top=.5in,left=.5in,right=.5in,bottom=.5in} }
	{\ifthenelse{ \lengthtest{ \paperwidth = 297mm}}
		{\geometry{top=1cm,left=1cm,right=1cm,bottom=1cm} }
		{\geometry{top=1cm,left=1cm,right=1cm,bottom=1cm} }
	}

% Turn off header and footer
\pagestyle{empty}
 

% Redefine section commands to use less space
\makeatletter
\renewcommand{\section}{\@startsection{section}{1}{0mm}%
                                {-1ex plus -.5ex minus -.2ex}%
                                {0.5ex plus .2ex}%x
                                {\normalfont\large\bfseries}}
\renewcommand{\subsection}{\@startsection{subsection}{2}{0mm}%
                                {-1explus -.5ex minus -.2ex}%
                                {0.5ex plus .2ex}%
                                {\normalfont\normalsize\bfseries}}
\renewcommand{\subsubsection}{\@startsection{subsubsection}{3}{0mm}%
                                {-1ex plus -.5ex minus -.2ex}%
                                {1ex plus .2ex}%
                                {\normalfont\small\bfseries}}
\makeatother

% Define BibTeX command
\def\BibTeX{{\rm B\kern-.05em{\sc i\kern-.025em b}\kern-.08em
    T\kern-.1667em\lower.7ex\hbox{E}\kern-.125emX}}

% Don't print section numbers
\setcounter{secnumdepth}{0}


\setlength{\parindent}{0pt}
\setlength{\parskip}{0pt plus 0.5ex}
\usepackage{amsfonts}
\usepackage{mathrsfs}
\usepackage[intlimits]{amsmath}
\usepackage{stmaryrd}
\usepackage{relsize}
\usepackage{etoolbox}
% \renewcommand*\ttdefault{cmvtt}
% \renewcommand*\familydefault{\ttdefault} %% Only if the base font of the document is to be typewriter style
% \usepackage[OT1]{fontenc}
\usepackage[english]{babel}
\usepackage[shortlabels]{enumitem}
\usepackage{mathtools}
\usepackage{amssymb}
\usepackage{stmaryrd}
\usepackage{cancel}
\usepackage{mdframed}
\usepackage{framed}{}
\usepackage{tablefootnote} 
\usepackage{listings}
\usepackage{amsthm}
\usepackage[dvipsnames]{xcolor}
\usepackage{etoolbox}
\usepackage[all]{xy}
\usepackage{tikz}
\usepackage{thmtools}
%\usepackage{mathpazo}
\usepackage{stmaryrd}
%\usepackage{stix}
%\let\Hermaphrodite\relax
\usepackage{mathabx}
\let\Sun\relax
\let\Moon\relax
\let\Mercury\relax
\let\Venus\relax
\let\Earth\relax
\let\Mars\relax
\let\Jupiter\relax
\let\Saturn\relax
\let\Uranus\relax
\let\Neptune\relax
\let\Pluto\relax
\let\Gemini\relax
\let\Leo\relax
\let\Libra\relax
\let\Scorpio\relax
\let\Aries\relax
\let\Taurus\relax
\usepackage{marvosym}
\usepackage[
   pdfpagelabels=true,
   pdftitle={Geometry and Topology for Data Analysis},
   pdfauthor={Min Tang},
 ]{hyperref}
\usepackage{bookmark}
\usepackage[usenames,dvipsnames]{pstricks}
\usepackage{epsfig}
\usepackage{pst-grad} % For gradients
\usepackage{pst-plot} % For axes
\usepackage[space]{grffile} % For spaces in paths
\usepackage{etoolbox} % For spaces in paths
\makeatletter % For spaces in paths
\patchcmd\Gread@eps{\@inputcheck#1 }{\@inputcheck"#1"\relax}{}{}
\makeatother

\usetikzlibrary{cd}
\usetikzlibrary{calc}
\theoremstyle{definition}
\newtheorem{definition}{Def}
\newtheorem{defi}[definition]{Def}
\newtheorem{problem}[definition]{Problem}
\newtheorem{example}[definition]{Exp}
\newtheorem{cor}[definition]{Cor}
\newtheorem{alg}[definition]{Alg}

\theoremstyle{theorem}
\newtheorem{theorem}[definition]{Thm}
\newtheorem{prop}[definition]{Prop}
\newtheorem{lemma}[definition]{Lem}
%\newtheorem{claim}[definition]{Claim}
%\newtheorem{corollary}[definition]{Corollary}
%\theoremstyle{definition}
\newtheorem{rem}[definition]{Rem}
\newtheorem{remark}[definition]{Rem}

%\AfterEndEnvironment{definition}{\noindent\ignorespaces}
%\AfterEndEnvironment{example}{\noindent\ignorespaces}
%\AfterEndEnvironment{theorem}{\noindent\ignorespaces}
%%\AfterEndEnvironment{satz}{\noindent\ignorespaces}
%\AfterEndEnvironment{corollary}{\noindent\ignorespaces}
%\AfterEndEnvironment{remark}{\noindent\ignorespaces}
%\AfterEndEnvironment{remark'}{\noindent\ignorespaces}
%\AfterEndEnvironment{proposition}{\noindent\ignorespaces}
%\AfterEndEnvironment{proof}{\noindent\ignorespaces}
\let\existstemp\exists
\let\foralltemp\forall
\newcommand{\tikzmark}[1]{\tikz[overlay,remember picture] \node (#1) {};}
\newcommand{\vsubset}{\rotatebox[origin=c]{90}{$\subset$}}
\newcommand{\vphi}{\phi}
\newcommand{\ol}{\overline}
%Differentiation
\newcommand{\D}{\, \mathrm{d} }
%Bold Symbols
\newcommand{\R}{\mathbb{R}}
\newcommand{\C}{\mathbb{C}}
\newcommand{\N}{\mathbb{N}}
\newcommand{\Q}{\mathbb{Q}}
%Calligraphic Symbols
\newcommand{\ZZ}{\mathcal{Z}}
\newcommand{\rspace}{\vspace{-1.1pc} }
\newcommand{\II}{\mathcal{I}}
\newcommand{\FF}{\mathcal{F}}
\newcommand{\QQ}{\mathcal{Q}}
\newcommand{\EE}{\mathcal{E}}
\newcommand{\PP}{\mathcal{P}}
\newcommand{\TT}{\mathcal{T}}
\newcommand{\MM}{\mathcal{M}}
\newcommand{\HH}{\mathscr{H}}
\newcommand{\RR}{\mathscr{R}}
\newcommand{\BB}{\mathscr{B}}
\newcommand{\CC}{\mathscr{C}}
\newcommand{\DD}{\mathscr{D}}
\newcommand{\LL}{\mathscr{L}}
\newcommand{\NN}{\mathscr{N}}
\renewcommand{\AA}{\mathscr{A}}
\newcommand{\myfont}{\fontfamily{ptm}\selectfont }
% Operator names
\newcommand{\id}{\operatorname{id}}
\newcommand{\Hom}{\operatorname{Hom}}
\newcommand{\del}{\partial}
\newcommand{\GL}{\operatorname{GL}}
\newcommand{\vol}{\operatorname{vol}}
\newcommand{\Var}{\operatorname{Var}} 
\newcommand{\Cov}{\operatorname{Cov}}
\newcommand{\End}{\operatorname{End}}
\newcommand{\SL}{\operatorname{SL}}
\newcommand{\Aff}{\operatorname{Aff}}
\newcommand{\Isom}{\operatorname{Isom}}
\newcommand{\Trans}{\operatorname{Trans}}
\newcommand{\Bild}{\begin{tiny}(Bild hier)\end{tiny}}
\newcommand{\Ueb}{\begin{tiny}\textbf{(Ü)}\end{tiny}}
\newcommand{\te}{\text}
\newcommand{\comment}[1]{}
\newcommand{\drawaline}{\rule{0.325\textwidth}{0.2pt}}
%\renewcommand{\def}{\definition}
\renewcommand{\ker}{\operatorname{ker}}
\renewcommand*{\forall}{\foralltemp\mkern2mu}
\renewcommand{\emptyset}{\varnothing}
\renewcommand{\Re}{\operatorname{Re}}
\renewcommand{\O}{\operatorname{O}}
\renewcommand{\Im}{\operatorname{im}}
%\renewcommand{\qedsymbol}{$\blacksquare$}
\renewcommand{\phi}{\varphi}
\makeatletter 
\AfterEndEnvironment{mdframed}{%
 \tfn@tablefootnoteprintout% 
 \gdef\tfn@fnt{0}% 
}
\DeclareMathOperator{\vertex}{Vert}
\usepackage{xpatch}
\makeatletter
%\AtBeginDocument{\xpatchcmd{\@thm}{\thm@headpunct{.}}{\thm@headpunct{}}{}{}}
%\numberwithin{equation}{section}
%\numberwithin{equation}{section}
%\numberwithin{definition}{section}
%\numberwithin{theorem}{section}
%\numberwithin{lemma}{section}
%\numberwithin{prop}{section}
%\numberwithin{corollary}{section}
%\numberwithin{problem}{section}
%\numberwithin{example}{section}
%\numberwithin{remark}{section}
%\numberwithin{claim}{section}
\setlist[enumerate,1]{label={(\roman*)}}
\linespread{1.1}

% -----------------------------------------------------------------------
